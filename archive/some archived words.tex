
\subsection{Robustness Check} \label{subsec:robustness_check}


\begin{table}[H]
  \begin{center}
    \begin{scriptsize}
    \caption{Robustness Check Using HHI Index}
    \label{tab:sylized_fact}
    {
\def\sym#1{\ifmmode^{#1}\else\(^{#1}\)\fi}
\begin{tabular}{l*{6}{c}}
\toprule
            &\multicolumn{1}{c}{(1)}&\multicolumn{1}{c}{(2)}&\multicolumn{1}{c}{(3)}&\multicolumn{1}{c}{(4)}&\multicolumn{1}{c}{(5)}&\multicolumn{1}{c}{(6)}\\
            &\multicolumn{1}{c}{log(income)}&\multicolumn{1}{c}{log(income)}&\multicolumn{1}{c}{price concession}&\multicolumn{1}{c}{price concession}&\multicolumn{1}{c}{log(lead times)}&\multicolumn{1}{c}{log(lead times)}\\
\midrule
L.ln\_income &     -0.0864\sym{***}&      -0.108\sym{***}&                     &                     &                     &                     \\
            &   (0.00562)         &   (0.00684)         &                     &                     &                     &                     \\
\addlinespace
yearx2\_density&       0.726\sym{***}&       0.136\sym{***}&     0.00260         &    -0.00141         &      0.0300         &      0.0299         \\
            &     (0.110)         &    (0.0438)         &   (0.00446)         &   (0.00156)         &    (0.0847)         &    (0.0317)         \\
\addlinespace
yearx3\_density&       0.430\sym{***}&       0.106\sym{***}&     0.00651         &    -0.00190         &       0.238\sym{***}&      0.0571\sym{*}  \\
            &     (0.113)         &    (0.0396)         &   (0.00397)         &   (0.00149)         &    (0.0849)         &    (0.0297)         \\
\addlinespace
yearx4\_density&       0.281\sym{***}&      0.0229         &     0.00635\sym{*}  &   -0.000630         &       0.163\sym{*}  &      0.0182         \\
            &     (0.103)         &    (0.0399)         &   (0.00363)         &   (0.00143)         &    (0.0850)         &    (0.0303)         \\
\addlinespace
yearx5\_density&       0.220\sym{**} &      0.0420         &    0.000861         &    -0.00108         &       0.215\sym{***}&      0.0531\sym{*}  \\
            &    (0.0976)         &    (0.0418)         &   (0.00324)         &   (0.00136)         &    (0.0716)         &    (0.0304)         \\
\addlinespace
yearx6\_density&      -0.273\sym{**} &      -0.191\sym{***}&    -0.00421         &    -0.00590\sym{**} &      0.0668         &     0.00649         \\
            &     (0.134)         &    (0.0662)         &   (0.00466)         &   (0.00256)         &     (0.102)         &    (0.0480)         \\
\addlinespace
yearx7\_density&      -0.358\sym{***}&      -0.133\sym{*}  &    -0.00135         &    -0.00796\sym{***}&       0.148         &      0.0297         \\
            &     (0.131)         &    (0.0767)         &   (0.00506)         &   (0.00299)         &     (0.112)         &    (0.0604)         \\
\addlinespace
broker\_410  &     0.00129         &     0.00610\sym{*}  &   0.0000350         &   0.0000271         &    0.000518         &     0.00487\sym{**} \\
            &   (0.00128)         &   (0.00342)         & (0.0000461)         &  (0.000137)         &  (0.000952)         &   (0.00246)         \\
\addlinespace
ln\_end\_price&       0.898\sym{***}&       0.966\sym{***}&      0.0651\sym{***}&      0.0787\sym{***}&       0.241\sym{***}&       0.269\sym{***}\\
            &    (0.0366)         &    (0.0591)         &   (0.00248)         &   (0.00387)         &    (0.0294)         &    (0.0434)         \\
\addlinespace
ln\_watch\_people&      0.0605\sym{***}&      0.0636\sym{***}&     0.00194\sym{***}&     0.00153\sym{***}&       0.332\sym{***}&       0.315\sym{***}\\
            &   (0.00473)         &   (0.00666)         &  (0.000208)         &  (0.000272)         &   (0.00530)         &   (0.00676)         \\
\addlinespace
ln\_watch\_time&      0.0303\sym{***}&      0.0308\sym{***}&   -0.000105         &   -0.000319\sym{*}  &      0.0266\sym{***}&      0.0450\sym{***}\\
            &   (0.00300)         &   (0.00409)         &  (0.000123)         &  (0.000174)         &   (0.00263)         &   (0.00342)         \\
\addlinespace
ln\_nego\_changes&      0.0170\sym{**} &      0.0299\sym{**} &     0.00178\sym{***}&     0.00258\sym{***}&       0.134\sym{***}&       0.134\sym{***}\\
            &   (0.00806)         &    (0.0122)         &  (0.000289)         &  (0.000459)         &   (0.00870)         &    (0.0102)         \\
\addlinespace
ln\_negotiation\_period&      0.0606\sym{***}&      0.0575\sym{***}&    -0.00184\sym{***}&    -0.00182\sym{***}&       0.116\sym{***}&       0.143\sym{***}\\
            &   (0.00445)         &   (0.00680)         &  (0.000194)         &  (0.000301)         &   (0.00487)         &   (0.00668)         \\
\addlinespace
L.price\_concession&                     &                     &      -0.183\sym{***}&      -0.205\sym{***}&                     &                     \\
            &                     &                     &   (0.00582)         &   (0.00744)         &                     &                     \\
\addlinespace
L.ln\_lead   &                     &                     &                     &                     &      -0.112\sym{***}&      -0.120\sym{***}\\
            &                     &                     &                     &                     &   (0.00457)         &   (0.00605)         \\
\midrule
\(N\)       &       80476         &       45060         &       77780         &       43578         &       80476         &       45060         \\
R-squared   &       0.892         &       0.908         &       0.638         &       0.699         &       0.925         &       0.929         \\
\bottomrule
\multicolumn{7}{l}{\footnotesize Standard errors in parentheses}\\
\multicolumn{7}{l}{\footnotesize \sym{*} \(p<0.1\), \sym{**} \(p<0.05\), \sym{***} \(p<0.01\)}\\
\end{tabular}
}
  
  
    \end{scriptsize}
  \end{center}
\end{table}

\begin{table}[H]
  \begin{center}
    \begin{scriptsize}
    \caption{Robustness Check}
    \label{tab:sylized_fact}
    {
\def\sym#1{\ifmmode^{#1}\else\(^{#1}\)\fi}
\begin{tabular}{l*{4}{c}}
\toprule
            &\multicolumn{1}{c}{(1)}&\multicolumn{1}{c}{(2)}&\multicolumn{1}{c}{(3)}&\multicolumn{1}{c}{(4)}\\
            &\multicolumn{1}{c}{log(income) [lower]}&\multicolumn{1}{c}{log(income) [higher]}&\multicolumn{1}{c}{log(lead times) [lower]}&\multicolumn{1}{c}{log(lead times) [higher]}\\
\midrule
L.ln\_income &     -0.0529\sym{***}&     -0.0929\sym{***}&                     &                     \\
            &   (0.00594)         &   (0.00539)         &                     &                     \\
\addlinespace
yearx2\_density&       0.197\sym{***}&      0.0979\sym{*}  &      0.0260         &      0.0582         \\
            &    (0.0586)         &    (0.0502)         &    (0.0406)         &    (0.0373)         \\
\addlinespace
yearx3\_density&       0.207\sym{***}&      0.0242         &       0.129\sym{***}&      0.0224         \\
            &    (0.0483)         &    (0.0463)         &    (0.0345)         &    (0.0373)         \\
\addlinespace
yearx4\_density&       0.114\sym{**} &     -0.0313         &      0.0540\sym{*}  &      0.0282         \\
            &    (0.0453)         &    (0.0440)         &    (0.0318)         &    (0.0326)         \\
\addlinespace
yearx5\_density&       0.128\sym{***}&      0.0437         &      0.0250         &      0.0647\sym{*}  \\
            &    (0.0423)         &    (0.0492)         &    (0.0315)         &    (0.0330)         \\
\addlinespace
yearx6\_density&     -0.0765         &      -0.237\sym{***}&      0.0139         &     -0.0107         \\
            &    (0.0814)         &    (0.0805)         &    (0.0493)         &    (0.0565)         \\
\addlinespace
yearx7\_density&      -0.116         &      -0.103         &     -0.0161         &      0.0817         \\
            &    (0.0883)         &    (0.0872)         &    (0.0724)         &    (0.0774)         \\
\addlinespace
broker\_410  &     0.00416\sym{***}&   -0.000302         &    0.000511         &    0.000930         \\
            &   (0.00155)         &   (0.00142)         &   (0.00114)         &   (0.00110)         \\
\addlinespace
ln\_end\_price&       0.945\sym{***}&       0.928\sym{***}&       0.233\sym{***}&       0.255\sym{***}\\
            &    (0.0429)         &    (0.0409)         &    (0.0326)         &    (0.0322)         \\
\addlinespace
ln\_watch\_people&      0.0863\sym{***}&      0.0501\sym{***}&       0.327\sym{***}&       0.331\sym{***}\\
            &   (0.00529)         &   (0.00505)         &   (0.00550)         &   (0.00564)         \\
\addlinespace
ln\_watch\_time&      0.0282\sym{***}&      0.0327\sym{***}&      0.0309\sym{***}&      0.0334\sym{***}\\
            &   (0.00392)         &   (0.00310)         &   (0.00295)         &   (0.00276)         \\
\addlinespace
ln\_nego\_changes&      0.0259\sym{***}&      0.0200\sym{**} &       0.160\sym{***}&       0.110\sym{***}\\
            &   (0.00911)         &   (0.00886)         &   (0.00812)         &   (0.00901)         \\
\addlinespace
ln\_negotiation\_period&      0.0488\sym{***}&      0.0657\sym{***}&       0.123\sym{***}&       0.129\sym{***}\\
            &   (0.00517)         &   (0.00487)         &   (0.00487)         &   (0.00534)         \\
\addlinespace
L.ln\_lead   &                     &                     &     -0.0922\sym{***}&      -0.107\sym{***}\\
            &                     &                     &   (0.00461)         &   (0.00496)         \\
\midrule
\(N\)       &       66143         &       67756         &       66143         &       67756         \\
R-squared   &       0.883         &       0.897         &       0.917         &       0.927         \\
\bottomrule
\multicolumn{5}{l}{\footnotesize Standard errors in parentheses}\\
\multicolumn{5}{l}{\footnotesize \sym{*} \(p<0.1\), \sym{**} \(p<0.05\), \sym{***} \(p<0.01\)}\\
\end{tabular}
}
  
  
    \end{scriptsize}
  \end{center}
\end{table}




\newpage 
\section{Appendix One \label{sec:appendix:first}}
\renewcommand{\thetable}{B\arabic{table}}
\setcounter{table}{0}
\renewcommand{\thefigure}{B\arabic{figure}}
\setcounter{figure}{0}



% \input{fig_tex/fig_another_figure.tex}

\newpage
\section{Appendix Two
\label{sec:appendix:two}}
\renewcommand{\thetable}{C\arabic{table}}
\setcounter{table}{0}
\renewcommand{\thefigure}{C\arabic{figure}}
\setcounter{figure}{0}







When it comes to the estimation of the entry effect and its continuous effect, we use the following model:

\begin{equation}
  \begin{aligned}
    Y_{it} & = \beta_0 + \beta_1 Pre_3 + \beta_2 Pre_2 + \beta_3 Entry \\
    & + \beta_4 Post_1 + \beta_5 Post_2 + \beta_6 Post_3  + \bf{\alpha} \bf{X}_{it} + \tau_{it} + \mu_i + \epsilon_{it}.
  \end{aligned}
\end{equation}

where $Pre$ is the pre-period of lianjia's entry, $Entry$ is the entry period, and $Post$ is the post-period of lianjia's entry. All other model settings are the same as the previous model.

Finally, we construct a proxy variable to capture the continuous effect of the DBI. The model is as follows:

\begin{equation}
  \begin{aligned}
    Y_{it} & = \beta_0 + \beta_1 proxy\_entry + \beta_2 proxy\_pos\_1 \\
    & + \beta_3 proxy\_pos\_2 + \beta_4 proxy\_pos\_3 + \bf{\alpha} \bf{X}_{it} + \tau_{it} + \mu_i + \epsilon_{it}.
  \end{aligned}
\end{equation}

where $proxy\_entry$ is the product of the entry effect and the DBI, and $proxy\_pos\_i$ is the product of the post-period effect and the DBI. All other model settings are the same as the previous model. The results are shown in Appendix Table \ref{tab:Baseline} and Appendix Table \ref{tab:Dynamic}.

However, our estimates may not be asymptotically efficient due to the endogeneity of DBI and serial correlation issues. First, DBI is endogenous because the entry of lianjia is endogenous. This is due to the fact that lianjia will choose better locations with higher potential revenues in order to maximize profits. Second, the model suffers from serial correlation because our sample is during the boom period of China's real estate market, and as the number of transactions in the real estate market increases, so do the revenues and prices in the brokerage industry. In addition, lianjia tend to be concentrated in popular areas, which makes for greater growth and thus confounds our estimates. To solve this problem, we use a dynamic panel model for estimation. Dynamic panel models can capture the dynamics of the process by using the lagged dependent variable as a regression variable. This helps in understanding how the past values of the variables affect their current values. 

To estimate the dynamic panel model, we use the Arellano-Bond (1991) GMM estimator, which is a two-step estimator. The first step is to use the lagged dependent variable as an instrument for the current dependent variable, and the second step is to use the lagged residuals as an instrument for the current dependent variable. In this paper, we do not use system GMM estimator because the system GMM must assume that the lagged explanatory variables are independent of factors that do not vary over time in the region, and this clearly cannot hold in the present setting. The model is as follows: [\textbf{this should be a line that describes what measures that we use to construct the model}]

\begin{equation}
  \begin{aligned}
    \Delta Y_{it} & = \rho \Delta Y_{it-1} + \beta_1 \Delta Measure_{it} + \bf{\alpha} \Delta \bf{X}_{it} + \Delta \tau_{it} + \Delta \epsilon_{it}.
  \end{aligned}
\end{equation}

where $\Delta Y_{it}$ is the first difference of the dependent variable, $\Delta Measure_{it}$ is the first difference of the our three measurement methods, $\Delta \bf{X}_{it}$ is the first difference of the control variables, $\Delta \tau_{it}$ is the first difference of the time dummy variable interacting with the fixed effect of the business area, and $\Delta \epsilon_{it}$ is the first difference of the error term. The standard errors are clustered at the individual level. The results are shown in Appendix Table \ref{tab:GMM}.


For Table \ref{tab:Baseline}, the result reveals a significant correlation between the lianjia's local market power and lianjia's income in this given community. However, we discovered no substantial influence of lianjia's local market power on the housing prices or transaction periods in the same community. This could be interpreted to mean that whilst Lianjia's local market power could induce an increase in income, it does not necessarily grant them any significant pricing power nor expedite the transaction period. Therefore, we may deduce that the augmentation of Lianjia's income is primarily driven by an increase in the number of transactions, presumably due to more property listings by Lianjia within the community. This suggests that although lianjia can control more market share in the local market, it does not necessarily mean that it can have monopoly power in this given region. This is partially because the price of the house is determined by the market, and there are also some potential other brokerages that can enter the market. However, lianjia's local market power can still have a significant influence on the income because they can list more housings and have more transactions in the local market.

To further test the results, we first check if lianjia's entry effect and continuous effect is significant. For column 1, 3, and 5 of the Table \ref{tab:Dynamic}, the result reveals that before the lianjia's entry, the natural logarithm of income, housing price and transaction period are not significant. After the entry, we see that after lianjia's entry into this local market, lianjia can immediately increase their income and keep making significantly more income comparing with previous periods. Besides, we see that the entry do not significantly increase the housing price immediately but significantly increase housing price one year after but later on turning back to the normal pricing. Lastly, the lianjia's entry do not significantly lowers the transaction periods. This is consistent with our previous results that lianjia's entry can significantly increase the income but not the housing price and transaction periods in equilibrium.

Lastly, we use the third measurement to test after the lianjia's entry, the continuous effect of lianjia's local power of the local market. For column 2, 4, and 6 of the Table \ref{tab:Dynamic}, the result reveals that the continuous effect of lianjia's local power of the local market is significant for the income, housing price and transaction periods. This result indicates that in different period, the lianjia's local market power can have different influence on the market.












































% we can not say that it is a martingale process because the housing market is assymetric information and martingale process is a fair game.
\section{Theoretical Model} \label{sec:theoretical_model}

% we can make the number of stores to $n$ instead of $2$ to make the model more general.

 % we may assume the number of houses in each community follows a normal distribution to make the model can have some probability distribution.

 % I think this model does not evaculate the meaningful information from the paper, and the paper does not have anything interesting to say in this model.


We can try to decompose the 






In spired by the finding of the stylized fact, we propose a theoretical model to explain the mechanism behind the empirical results. Our model accurately reflects the real-world scenario in which sellers face uncertainty regarding the number of buyers each platform might attract, whereas buyers have the advantage of accessing comprehensive online information to make informed decisions about potential house purchases. This model also encapsulates the dynamics of a two-sided market, characterized by the mutual attraction between buyers and sellers. However, due to asymmetric information, sellers are unable to determine which brokerage will attract the most buyers, creating a gap in their market strategy effectiveness.

To characterize the real estate brokerage market, we can assume that there are $\mathcal{N}$ types of brokerages in the market, where the type $1$ denote the large real estate brokerage that is incorporating the online platform and the offline stores strategies, and the type $2$ to type $\mathcal{N}$ denote the small real estate brokerage that only have the offline stores and they should compete with the type $1$ stores and type $2, \ldots, \mathcal{N}$ stores in the market. Each type of brokerage follows a standard percentage charge by $r_j$ where $j \in \{1, 2, \ldots, \mathcal{N}\}$ and they can not manipulate the percentage charge in each submarket.

We can assume that there are $M$ number of communities in the market, and each market has $\kappa_i, i \in \{1, \ldots, M\}$ number of houses listed. Assume that the sellers are indifferent among the brokerages and therefore, their chossing is only dependent on the proportion of stores in the market. We can denote the number of two types of brokerages by $\mathcal{B}_{1, i}$ and $B_{2, i}$, respectively and therefore, the number of houses listed in the brokerage $j$ is $\kappa_{i} \times \frac{\mathcal{B}_{i, i}}{\mathcal{B}_{i, 1} + \mathcal{B}_{i, 2}}$. We can assume that the price of the house is determined by the hedonic pricing model from various confounding factors that are determined by the community and unique characteristics of the housing, and we can denote it as $P_i$ that can be ranked and distributed over the range $[\underline{P}_i, \overline{P}_i]$. Sellers have a reservation price $\tilde{P}_i$ over the house they sell.

The brokerages have the same positive cost function denoted as $C(\cdot)$ where $\frac{\partial C(\cdot)}{\partial \mathcal{B}_{j}} > 0$ and $\frac{\partial C^2(\cdot)}{\partial \mathcal{B}_{j}^2} > 0$ where $j \in \{1, 2, \ldots, \mathcal{N}\}$. The intuition is that because of the marginal cost of the platform is $0$, when considering the cost of opening additional stores in the market, the brokerage does not incur that cost into consideration and therefore, the cost function is equal for both types of stores. Moreover, the marginal cost of opening additional stores is increasing, which is consistent with the real estate brokerage market because the more stores the brokerage opens, the administrative cost will increase.

We can inherent from the previous literature that there are broad searchers and narrow searchers in the market, \citep{10.1257/aer.20141772}. Specifically, we can assume that there are $L^N_i$ narrow searches that are only searching for the community $i$ and let $L^B$ denote the set of broad searchers, defined as those individuals who consider a $\beta$ proportion of all available houses within their expansive search criteria. When $\beta = 1$ it means that they are searching for all the available houses in the whole city and $\beta < 1$ indicates a limited scope, possibly due to preferences or capacity constraints. Correspondingly, $L^B_i$ represents the subset of broad searchers within community $i$, and the expected number is given by $L^B_i = L^B \times \beta$. We can denote the buyers' utility of owning the house by $U_{i, j}$ 

We can first focus on the naive searchers in the real estate market. These individuals are typically drawn to platforms based on the breadth of listings available. Specifically, the probability that a naive searcher will choose brokerage $j$ can be characterized by the principle of the two-sided market. This decision is influenced by the ratio of the number of houses listed by brokerage $j$ to the total number of houses listed within community $i$. This model aligns with the framework proposed by \citep{Armstrong2006}, where market attractiveness is proportional to the available inventory a brokerage offers relative to the community aggregate:

\begin{equation}
  N_{i, j} = L^N_i \times \frac{\mathcal{B}_{i, j}}{\sum_{j^\prime=1}^{\mathcal{N}}\mathcal{B}_{i, j^\prime}}. \label{eq:naive_prob}
\end{equation}

However, the broad searchers in the market are attracted to the platform by the networking because they are searching for all the communities in the market, and buyers have different preferences for brokerages based on their online presence, reputation and other factors, with private information $\theta_{b, j}$ representing the buyer's preference for brokerage $j$.

\begin{equation}
  B_i = L^B_i \times \alpha_j \times \frac{\mathcal{B}_{i, j} \times \theta_{b, j}}{\sum_{j^\prime=1}^{\mathcal{N}}\mathcal{B}_{i, j^\prime} \times \theta_{b, j^\prime}} \label{eq:broad_prob}
\end{equation}

and we can release this assumption later to make the customers have private preference towards each type of brokerage and the brokerages can not observe this pattern. Combinning \eqref{eq:naive_prob} and \eqref{eq:broad_prob}, we can get the total number of customers in the community $i$ and the brokerage $j$ is:

\begin{equation}
  L_{i, j} = N_{i, j} + B_{i, j} = (L^N_i + L_B \times \beta \times \alpha_j) \times \frac{\mathcal{B}_{i, j} \times \theta_{b, j}}{\sum_{j^\prime=1}^{\mathcal{N}}\mathcal{B}_{i, j^\prime} \times \theta_{b, j^\prime}}
   \label{eq:total_prob}
\end{equation}

We can assume the probability of a successful match between buyers and sellers, which increases with the number of participants on the platform, as $\tau_{i, j} = f(\frac{\mathcal{B}_{i, j}}{\sum_{j^\prime=1}^{\mathcal{N}}\mathcal{B}_{i, j^\prime}})$ and the revenue for each brokerage $j$ in community $i$ can be calculated by considering the total number of transactions completed, which is the product of the number of customers handled by the brokerage and the probability of transaction completion.

\begin{equation}
  R_{i, j} = \tau_{i, j} \times L_{i, j} \times P_i \times r_j \label{eq:revenue}
\end{equation}

and the correspondingly, the total profit for the brokerage $j$ in the community $i$ is:

\begin{equation}
  \Pi_{i, j} = R_{i, j} - C(\mathcal{B}_{i, j}) \label{eq:profit}
\end{equation}

The consumer surplus is then: $CS = \sum_{i=1}^M \sum_{j=1}^{\mathcal{N}} \int_0^{L_{i, j}} (U_{i, j} - P_i) dx$.

% \begin{assumption}[homogenous brokerage]
%  The brokerage $1$'s network effect is greater than other brokerages in the market and assume that the small brokerages (Type $2$ to $\mathcal{N}$) are homogenous and they have the same network effect in the market, i.e., $\alpha_2 = \ldots = \alpha_{\mathcal{N}} = \alpha_s$, where $\alpha_s$ is the network effect for small brokerages. We have $\alpah_1 > \alpha_s$.
% \end{assumption}

\begin{proposition}[brokerage $1$'s optimal choice]
  In equilibrium, the market share of Brokerage 1 (Type 1) is greater than its proportion of the number of stores, and the optimal brokerage fee charged by Brokerage 1 is higher than that charged by other brokerages. 
\end{proposition}

\begin{proof}
  The total number of customers for Brokerage 1 in community $i$ is given by $L_{i, 1}$ and the total number of stores is $L_{i}$:

  \begin{equation}
    \text{Market Share}_{i, 1} = \frac{L_{i, 1}}{L_i}
  \end{equation}
  
  Since $\alpha_1 > \alpha_s$ and assuming $\theta_{b, 1} \geq \theta_{b, j}$ and Brokerage 1 will attract a higher proportion of broad searchers $L^B$, resulting in a larger market share. Thus, in equilibrium, the market share of Brokerage 1 is greater than its proportion of the number of stores.

  To maximize the profit, brokerage $1$ will change $r_1$ such that: $\frac{\partial \Pi_{i, 1}}{\partial r_1} = \frac{\partial R_{i, 1}}{\partial r_1} - \frac{\partial C(\mathcal{B}_{i, 1})}{\partial r_{1}} = 0$. Since $\tau_{i, 1}$ and $L_{i, 1}$ are higher for brokerage $1$ due to the stronger network effect, the optimal $r_1$ will be higher compared to the fees charged by other brokerages, $r_j$ for $j \neq 1$.
\end{proof}

In the following proposition, we will derive the total surplus of the model.

This model is not good right now because there is no better way to conduct welfare analysis.



% now consider the two-period entry model now consider the case of the first period there is an equilibrium with all other broekrages but in the next period there is an entry of the first brokerage then derive the new equilibrium of the market and conduct the wellfare analysis in this case

\begin{proposition}[two-period entry model]

\end{proposition}

% where $N^e(B_{1, i})$ is a function representing the network effects, and is modeled as $N^e(B_{1, i}) = 1 + \alpha \log(1 + B_{1, i})$ where $\alpha$ is a parameter that quantifies the strength of network effects and the logarithmic form captures diminishing returns to scale as the number of brokerages increases. The revenue for each brokerage type $j$ in community $i$ is proportional to the number of houses they sell, which depends on the number of customers they attract: 

% where $N_{i, j, t}$ is the number of customers selecting type $j$ brokerage in community $i$ at time period $t$, estimated by:

% \begin{equation}
%  N_{i, j, t} = \lambda_i p_{i, j, t}
%\end{equation}

%and $\lambda_i$ is the appropriate part of $\lambda$ that arrives in community $i$.


%give me a mathematical economics model characterizing the partial equilibrium of real estate broekrage and the competitive market where there are two types of brokerages denoted 1 if the brokerage is the large brokerage that it can communicate with other stores in different regions of the same type 1 brokerage, and denote 2 if it is small brokerage that it can not communicate with the other brokerages in the market

% The first proposition is that the charging fee should be higher comparing with type $2$ brokerage.

% The second proposition is that the number of stores in high popular regions should be higher in the number of stores and they are more likely to form a network effect.

% The third proposition is the introduction of the platform should increase the number of transactions and the number of stores in the market.

% The fourth proposition is that the entry of the large brokerage can increase the transaction number and the number of stores in the market.

% The fifth proposition should be the fact that when there is negative shock to the customers, the number of transactions and the number of stores should decrease and the number of brokerage fee should decrease in the long run (which can be modeled using graph or simulation).

% The sixth proposition should be the fact that before the increase of the 











We can examine the following two effects:

\begin{itemize}
  \item The real estate brokerage's network effect where the brokerages are more likely to form a edge between each other (this is demonstrated in the ACN's brokerage network model)
  \item The partial effect of the network effect is not characterized by the price level changes or the brokerage fee increase.
  \item The network effect should be demonstrated by the increase in the number of transactions and the number of the stores within given radius comparing with the total number of houses in this market.
\end{itemize}
